% Options for packages loaded elsewhere
\PassOptionsToPackage{unicode}{hyperref}
\PassOptionsToPackage{hyphens}{url}
%
\documentclass[
]{article}
\usepackage{amsmath,amssymb}
\usepackage{iftex}
\ifPDFTeX
  \usepackage[T1]{fontenc}
  \usepackage[utf8]{inputenc}
  \usepackage{textcomp} % provide euro and other symbols
\else % if luatex or xetex
  \usepackage{unicode-math} % this also loads fontspec
  \defaultfontfeatures{Scale=MatchLowercase}
  \defaultfontfeatures[\rmfamily]{Ligatures=TeX,Scale=1}
\fi
\usepackage{lmodern}
\ifPDFTeX\else
  % xetex/luatex font selection
\fi
% Use upquote if available, for straight quotes in verbatim environments
\IfFileExists{upquote.sty}{\usepackage{upquote}}{}
\IfFileExists{microtype.sty}{% use microtype if available
  \usepackage[]{microtype}
  \UseMicrotypeSet[protrusion]{basicmath} % disable protrusion for tt fonts
}{}
\makeatletter
\@ifundefined{KOMAClassName}{% if non-KOMA class
  \IfFileExists{parskip.sty}{%
    \usepackage{parskip}
  }{% else
    \setlength{\parindent}{0pt}
    \setlength{\parskip}{6pt plus 2pt minus 1pt}}
}{% if KOMA class
  \KOMAoptions{parskip=half}}
\makeatother
\usepackage{xcolor}
\usepackage{graphicx}
\makeatletter
\def\maxwidth{\ifdim\Gin@nat@width>\linewidth\linewidth\else\Gin@nat@width\fi}
\def\maxheight{\ifdim\Gin@nat@height>\textheight\textheight\else\Gin@nat@height\fi}
\makeatother
% Scale images if necessary, so that they will not overflow the page
% margins by default, and it is still possible to overwrite the defaults
% using explicit options in \includegraphics[width, height, ...]{}
\setkeys{Gin}{width=\maxwidth,height=\maxheight,keepaspectratio}
% Set default figure placement to htbp
\makeatletter
\def\fps@figure{htbp}
\makeatother
\setlength{\emergencystretch}{3em} % prevent overfull lines
\providecommand{\tightlist}{%
  \setlength{\itemsep}{0pt}\setlength{\parskip}{0pt}}
\setcounter{secnumdepth}{-\maxdimen} % remove section numbering
\ifLuaTeX
  \usepackage{selnolig}  % disable illegal ligatures
\fi
\IfFileExists{bookmark.sty}{\usepackage{bookmark}}{\usepackage{hyperref}}
\IfFileExists{xurl.sty}{\usepackage{xurl}}{} % add URL line breaks if available
\urlstyle{same}
\hypersetup{
  pdftitle={About us},
  hidelinks,
  pdfcreator={LaTeX via pandoc}}

\title{About us}
\author{}
\date{}

\begin{document}
\maketitle

\hypertarget{jackstraws-past-to-present}{%
\section{Jackstraws past to present}\label{jackstraws-past-to-present}}

Jackstraws Morris was founded in 1977 by former members of Updown Hill
Morris from Windlesham, and first danced out on Boxing Day of that year,
at the White Hart Inn in Pirbright. Until 1996 the side was based in the
village of Pirbright in Surrey. We moved to Farnborough in Hampshire
after the Red Cross hall in which we used to practise was demolished,
and then spent our Monday evenings honing our skills in a local school's
assembly hall. In 2006, we moved back to Surrey, and we now practise in
the Hale Institute Hall, Hale, near Farnham.

\hypertarget{the-early-years}{%
\subsection{The Early Years}\label{the-early-years}}

{[}By Jackie Weller (formally Emerson), a founder member and the first
Squire:{]}

Jackstraws Morris was formed in September 1977. The side was founded on
limited experience, bags of enthusiasm, complete dedication and no
money. The 12 founder members were as follows:

\includegraphics{/img/earlystraws.gif\#left} Jackie Emerson (Squire);
Sandy Grigson (Bag-person); Anne Manley (Treasurer);\\
Maggie Lawrance; Angela Rowe; Pip Paget; Jill Jackson; Jane Porter
(Fool);\\
Barbara Porter (Musician); Ivan North (Musician); Pete Rowe (Forman);
Pete Lister

The idea to form the side came from Jackie Emerson and Sandy Grigson,
who, although ex-members of Updown Hill, each had only one year of
dancing experience and no experience of traditional Cotswold Morris. We
considered ourselves very fortunate when we attracted the interest of
two Pilgrim Morris Men (Pete Rowe \& Pete Lister) who instructed us in
the traditions of Bampton and Bledington, which became the side's two
main traditions for a number of years. Pete Rowe gave Jackstraws an
excellent grounding but sadly left us after the first year due to
hostility from his own side. We were then instructed by Alan Dean and
John Glaister from Thames Valley Morris, both Morris Men of many years
standing, who were pleased to pass on their vast knowledge of the dance
to us until we felt confident enough to instruct from within the side,
without relying on outside help. It was based on their instruction of
the tradition that Carol Smith introduced the Fieldtown tradition when
she became foreman four years after the side was started.

Jackie Emerson had been the fool in the Updown Hill side and as she
modelled the side in her own image the idea of a smock for everyone was
unanimously agreed by all at the first Jackstraws meeting. The idea of a
Surrey Smock was suggested by Barbara Porter who informed us there was
an authentic one on display at the Weybridge Museum (and still is to
this day I think), and it was from there we obtained the pattern.

By Christmas of that year we acquired five other members, Fran Flint,
Carol Smith, Chris Mort (later English), Shanie Hockey and Roger
English. Two future Squires in Carol \& Chris and a future Bag-person in
Fran, and they completed the Jackstraws line up in the first year
1977/78 . The photograph shows most of those original members. It was
taken during Jackstraws' first year.

As I hope you can see from the picture, Jackstraws in the beginning had
a Hobby Horse. Daisy (the cow) came along in our third year. The
original Daisy was stolen from us whilst we were dancing at a pub in
Thorpe. The replacement, the second Daisy, was too unwieldy and a third
Daisy was constructed in 1987. Jackstraws Morris first danced out as a
side on December 26th (Boxing Day)1977. The following summer we had a
full programme starting with May Day (all day \& evening) the usual
peppering of fetes, ceilidhs and pubs, hosted a Day of Dance, took the
side to an enormous rally in London hosted by `The Women's Morris
Federation' (as it was called then) and danced brilliantly in Trafalgar
Square supported by Fool \& Hobby Horse, and the full side went down to
Devon for the Sidmouth Folk Festival for the whole week and danced every
day. Not bad for a side in it's novice year.

The Jackstraws Day of Dance for years was always held in September. The
first Day of Dance was held on September 10th to mark their first
Squire's Birthday and was always held afterwards on the Saturday nearest
to that date.

\hypertarget{our-practice-and-post-practice-venues}{%
\subsection{Our practice (and post-practice)
venues}\label{our-practice-and-post-practice-venues}}

Morris teams are often strongly associated with a location or area of
the country. Jackstraws have always been based in the Surrey/Hampshire
borders.

The team was originally a break-away from Updown Hill Morris, based in
Windlesham near Bagshot, and many of the members lived in that area.

Mondays were chosen as the practice night, due to members' commitments
on other nights and a suitable hall was found in the picturesque village
of Pirbright on the army ranges north of Aldershot. The Red Cross Hall
on Dawney Hill was a reasonable size -- we even held our annual barn
dance there, but it has to be said it was very cosy! At one point we had
so many dancers coming to practice we felt we needed a bigger hall. We
tried a hall in West End, to the west of Woking, but the acoustics
weren't great, and we returned to Pirbright.

Our `local', the pub we retired to after practice, was the Cricketers,
on the other side of the green. We had initially patronised the (rather
more posh) White Hart, on the corner of the green. Our debut performance
had taken place there on Boxing Day 1977, but we stopped going there
when they refused entry to Fran and Bob's friend Ian Anderson on the
grounds that he was wearing a leather jacket and therefore lowering the
tone!

Dancing to greet the sunrise on May morning was introduced in the early
1980s, and we danced each year on Pirbright Green, where we wouldn't be
disturbing anybody -- very rarely had an audience, but once entertained
a camel when a circus was pitched on the green!

We were based in Pirbright until 1996, when the hall was sold and the
site re-developed. We then tried the Pirbright scout hut on School Lane,
but the floor was alarmingly flexible, so we looked elsewhere.

One of our dancers, Maggie Bonfield, was the Head Teacher at Farnborough
Grange School, in Moor Road, Farnborough (in the shadow of the M3
motorway), and we practiced in the school hall for the next 10 years. We
tried several of the nearby pubs for post-practice relaxation, including
the Thatched Cottage, the George (since sadly demolished) and the Ship.
The latter was the venue for our 21st anniversary Boxing Day dance-out
in 1997, but on many Boxing Days we danced at Kris's local, the Crown at
Badshot Lea, a friendly village pub situated in a cul-de sac (so dancing
on the road did not inconvenience much traffic).

Dancing at dawn on May Day took place at the pond outside the Community
Centre in Farnborough, and we sometimes had people watching as they went
to the Leisure Centre. The tradition died when we started to become low
on numbers -- not enough dancers crazy enough to turn out at such a
time.

We moved back into Surrey in 2006 (?), basing ourselves at the Hale
Institute in Upper Hale, just north of Farnham. The pub just down the
road, the Ball and Wicket, brewed its own beer, but the beer wasn't
great and we never felt particularly welcome. When Judy had a very upset
tummy after sampling one of their brews, we decided to look elsewhere.
How glad we were to discover the Alfred, a wonderful real ale pub down
in the lanes of Upper Hale, run by the friendly Curran family. Although
they had generally closed at 9 on Mondays, they remained open specially
for us -- initially if we telephoned in advance, but later they just
changed the opening hours. We held our AGMs there, our Boxing Day
dance-outs were there (of course) and included the mummers play inside
the crowded pub\ldots{} The Covid pandemic put paid to the fun, though.
No dancing out happening in 2020, and only a little in 2021 -- and our
planned Boxing Day at the Alfred had to be cancelled when Shirley and
Charlotte had to isolate.

On return from the period of lockdowns to control the spread of Covid,
we found the Alfred was not opening at all on Mondays. The Ball and
Wicket had been closed and converted into housing, so we turned to the
Shepherd and Flock, on the `Shepherd and Flock roundabout' to the east
of Farnham, for our post-practice relaxation. The Alfred still warmly
welcomed the Boxing Day dancing and mumming, and we still make a point
of inviting Pilgrim Morris to dance with us there on a Wednesday night
in the summer.

\hypertarget{our-kit}{%
\subsection{Our Kit}\label{our-kit}}

\begin{figure}
\centering
\includegraphics{/img/oldkit.gif\#left}
\caption{Old kit}
\end{figure}

\begin{figure}
\centering
\includegraphics{/img/newkit.jpg\#right-190}
\caption{New kit}
\end{figure}

Originally a white, hand made smock with green smocking was worn, based
on a traditional Surrey smock design. Jackstraws was one of the first
revival women's teams to wear trousers (in 1977!). The kit was completed
with a green neckerchief, white shoes, bell pads with green and white
ribbons and a white hat decorated with flowers.

In 2005 we changed our kit to something more modern. The new kit retains
the same colours as the old kit (i.e.~green and white), but instead of
smocks we wear loose white shirts, with ribbons attached at the
shoulders, and we have given up the flowered hats.

\hypertarget{daisy}{%
\subsection{Daisy}\label{daisy}}

\includegraphics{/img/Daisy.jpg\#right-240} On occasions Daisy, a cow,
appears with Jackstraws. Daisy was specially made in 1987, our 10th
Anniversary year. She is a papier-mache carnival head with a brown
spotted cloth for a body. She used to have a shy and retiring
personality which kept her on the side-lines, entertaining the children
in the audience, but since appearing at the Warwick Folk Festival in
2007, she has been an outrageous scene-stealer.

\hypertarget{our-21st-birthday}{%
\subsection{Our 21st Birthday}\label{our-21st-birthday}}

{[}By Sue Hamer-Moss, Bagperson{]}

In 1999 Jackstraws were 21 years old and we celebrated in style on 20th
June in Farnham, Surrey. Braving the fortunately very short-lived
drizzle, Jackstraws and 6 invited guest sides celebrated with generous
amounts of Bucks Fizz in scenic Farnham Castle before dancing their way
around Farnham for the day -- not too far away from the many
hostelries!! A pageantry of colour, music, dancing styles -- Cotswold,
Border and North West -- not to mention strange mascots, greeted the
citizens of Farnham as we progressed around the town.

The evening started with a slap-up buffet meal and a suitably adorned
birthday cake, followed by a ceilidh with a great band -- Junction 24 --
with Hugh Ripon calling. It was wonderful to see so many past members of
the side turning up to celebrate, including the very first squire of
Jackstraws, from 21 years ago. Buy Cipro. Party spots included
Jackstraws dancing a blind stick dance- fortunately using sausage
balloons (the most difficult bit was blowing them up!) and Fleet Morris
doing a hilarious spoof of Ring of Bells dressed as commuters on our
favourite crowded service.

Our thanks to our guests -- Alton, Fleet, Pigsty, Rose and Castle,
Windsor and Yateley and we look forward to the next 21 years!

\hypertarget{jackstraws-at-40}{%
\subsection{Jackstraws at 40}\label{jackstraws-at-40}}

\includegraphics{/img/cake.jpg\#left-140} In 2018 we celebrated our 40th
anniversary, enjoying a Day of Dance in our home town of Farnham.
Joining us on the day were our friends from Hook Eagle Border Morris,
Pilgrim Morris Men, Fleur de Lys Morris and Fleet Morris. Festivities
with tea, cake and beer after the dancing was concluded.

\hypertarget{the-future}{%
\subsection{The Future}\label{the-future}}

We are proud of our history our high standards of dance and performance
-- to help us continue we need new members to join our friendly team --
contact us to find out more.

\end{document}
